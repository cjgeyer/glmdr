
\documentclass[11pt]{article}

\usepackage{natbib}

\begin{document}

\title{Design Document for GLMDR Package}

\author{Charles J. Geyer}

\maketitle

We implement the stuff in \citet{geyer-gdor}, fixing it where necessary.
We start with an R function just like the R function \texttt{glm} in
the R core, except for the following differences
\begin{itemize}
\item It does solutions at infinity right!
\item It only does exponential family models.  No quasi-likelihood.
    No non-canonical link functions.  No non-discrete distributions.
    That leaves only \texttt{binomial("logit")} and \texttt{poisson("log")}
    of the families allowed for the R function \texttt{glm}.
\item We might also want to do zero-inflated Poisson regression using
    the aster parameterization so it is exponential family too.
    But ignore this for the first version.  It would complicate things
    to allow aster graph models, even one this simple.
\item We certainly want to do multinomial and product multinomial response.
    Unlike the approaches of the R functions \texttt{multinom} in the R
    ``recommended'' (meaning in every R installation) packages \texttt{mgcv}
    and \texttt{nnet} we don't have a special function to do that.
    We just allow conditioning on certain marginals.
    We condition on $M^T y$, where $M$ is the model matrix for a certain
    conditioning formula.  If the conditioning formula is \verb@~ 1@, then
    we are doing multinomial regression.  If it is a more complicated formula,
    then product multinomial.
\item In order for the latter to work correctly, we also have to handle
    directions of constancy correctly.
\end{itemize}
The R core has the following methods for objects of class \texttt{"glm"}
as of this writing (R-3.2.4).
\begin{verbatim}
add1.glm
anova.glm
confint.glm
cooks.distance.glm
deviance.glm
drop1.glm
effects.glm
extractAIC.glm
family.glm
formula.glm
influence.glm
is.glm
logLik.glm
model.frame.glm
nobs.glm
predict.glm
print.glm
print.summary.glm
residuals.glm
rstandard.glm
rstudent.glm
summary.glm
vcov.glm
vcov.summary.glm
weights.glm
\end{verbatim}
We definitely need the corresponding
\begin{verbatim}
anova.glmdr
confint.glmdr
extractAIC.glmdr
predict.glmdr
summary.glmdr
print.summary.glmdr
\end{verbatim}
and maybe need
\begin{verbatim}
add1.glmdr
drop1.glmdr
vcov.glmdr
vcov.summary.glmdr
\end{verbatim}

Start with the main model fitting function.  The signature for the R
function \texttt{glm} is
\begin{verbatim}
glm(formula, family = gaussian, data, weights, subset, 
    na.action, start = NULL, etastart, mustart, offset,
    control = list(...), model = TRUE, method = "glm.fit",
    x = FALSE, y = TRUE, contrasts = NULL, ...) 
\end{verbatim}
How much of that do we want?  We don't need \texttt{weights}.
We don't need \texttt{method}.  I guess we allow \texttt{contrasts}.
I have never used that, but it seems usable.  For family, we only
allow \texttt{"binomial"} or \texttt{"poisson"}.  Nothing else is
relevant.

Thus our signature seems to be
\begin{verbatim}
glmdr(formula, conditioner = NULL, family = c("poisson", "binomial"),
    data, subset, na.action, start = NULL, etastart, mustart,
    offset, control = list(...), model = TRUE,
    x = FALSE, y = TRUE, contrasts = NULL, ...) 
\end{verbatim}

\begin{thebibliography}{}

\bibitem[Geyer(2009b)]{geyer-gdor}
Geyer, Charles J. (2009b).
\newblock Likelihood inference in exponential families and directions
    of recession.
\newblock \emph{Electronic Journal of Statistics}, \textbf{3}, 259--289.

\bibitem[Geyer, et~al.(2007)Geyer, Wagenius, and Shaw]{geyer-wagenius-shaw}
Geyer, C.~J., Wagenius, S., and Shaw, R.~G. (2007).
\newblock Aster models for life history analysis.
\newblock \emph{Biometrika}, \textbf{94}, 415--426.

\end{thebibliography}

\end{document}

